%!TeX encoding = UTF-8
%!TeX root = ../main.tex
\chapter{Architecture (hardware and software)}
\label{chapter:architecture}

This chapter explains the general architecture of the project, on both hardware and software points of view.

\section{Hardware architecture}

The eco-classifier project has the following hardware parts:

\begin{itemize}
    \item One front-facing USB3 camera
    \item One top-facing USB3 camera
    \item Physical cables to connect those to the main PC
    \item A light panel
    \item A main analysis PC (which is where most of this project should be running)
    \item An ethernet cable connecting the main PC to the Profinet-enabled automate
    \item An outbound (internet) link?
\end{itemize}

Nothing more.

\todo{Describe the cameras, specifications, etc}

\section{Software architecture}

The architecture is different for each part of the project, namely:

\begin{itemize}
    \item Image grabbing and capturing (for dataset-building phase)
    \item Deep-Learning computation (for training phase)
    \item Inference (for inference phase)
\end{itemize}

This repository contains all the required code for the three phases, but this code cannot execute
on the same machine at each step.

Some code require hardware parts to run (dataset building and inference need the cameras), some don't
(deep-learning training doesn't require anything else but a powerful GPU).

The following section is thus split in those different phases.

\subsection{Image grabbing}

The image grabbing part consists in a software piece that reads from (both) cameras and record enough images to
allow a careful analysis of the captured images.

Capture must be as close as possible to the real conditions of the project.

Capture must retain as much elements as possible but not overflow the hard drives.

Capture is made through an OpenCV + Basler software combination. The general flow is:

\begin{enumerate}
    \item Open cameras, start acquiring (without saving on disk)
    \item If "something moves" on camera, start the recording process
    \item Record as much frames as possible to ensure high resolution acquisition
    \item Close camera (and start over)
    \item Upload material on an Azure blob
\end{enumerate}

This piece of software must be installed on a temporary PC that will be plugged to the cameras.

The software must start when the PC starts (to handle reboots)

The acquired images must be either uploaded or accessed remotely. Thus, the PC needs a permanent or semi-permanent internet connexion.

\subsubsection{Requirements}

\begin{itemize}
    \item Python >= 3.6
    \item OpenCV
    \item Snap7 software (for \gls{plc} communication)
    \item Basler software (bundled!)
    \item Semi-async mode (main loop for image capture, async mode for the upload part)
\end{itemize}

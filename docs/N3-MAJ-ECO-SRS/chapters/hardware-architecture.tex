%!TeX encoding = UTF-8
%!TeX root = ../main.tex
\chapter{Architecture (hardware)}
\label{chapter:hardwarearchitecture}

This chapter explains the general architecture of the project (hardware).

% \section{Hardware architecture}

The eco-classifier project has the following hardware parts:

\begin{itemize}
    \item One front-facing GigE camera
    \item One top-facing GigE camera
    \item Physical cables to connect those to the main PC
    \item A light panel, connected to one of the cameras
    \item An "ambiant light panel", always on for computer vision operations
    \item A main analysis PC (which is where most of this project should be running)
    \item An ethernet cable connecting the main PC to the Profinet-enabled automate
    \item An outbound (internet) link (out of scope)
\end{itemize}

Nothing more.

\todo{Describe the cameras, specifications, etc}

\section{PC}

Here are our requirements for the PC connected to both the \gls{plc} and the cameras.

\begin{itemize}
    \item At least 16Gb RAM. The images are huge and both the images and the algorithm need to fit into memory
    \item An Intel \texttt{core i5} or \texttt{core i7} CPU
    \item No GPU necessary
    \item At least 3\footnote{They often come by 4} dedicated Gigabit ethernet ports
    \item DIN rail physical shape
    \item Maximum DIN module width has to be validated with \gls{dinatec}
    \item 24~Volts power
\end{itemize}

\section{Cameras and lenses}

Both cameras are of the same type:

\begin{itemize}
    \item \texttt{Basler acA5472-5gc} area scan cameras
    \item Ethernet (GigE) connectivity
\end{itemize}

Both lenses are of the same type:

\begin{itemize}
    \item 25mm 1.4
    \item C-mount
\end{itemize}

In this document, they are called \gls{vtcamera} and \gls{hzcamera}.


\section{Light panel}

Provided by i2s, out of scope for now.

\section{Networking}

Networking is critical in our application. Several issues must be considered.

\begin{itemize}
    \item Use a 1Gb switch/router, or, even better, dedicated Ethernet ports on the PC
    \item Use Jumbo Frames (see \url{https://linuxconfig.org/how-to-enable-jumbo-frames-in-linux})
    \item 9500~packet delay seems to be the accurate value (see )
\end{itemize}

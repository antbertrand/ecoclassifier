%!TeX encoding = UTF-8
%!TeX root = ../main.tex
\chapter{Software specifications}
\label{chapter:softwarereqs}

\section{Bootstrap module}

The bootstrap module must initialize the machine and must be able to self-update from the internet.

With minimal manual operations, a PC have to be setup and the required software has to be installed on it.

A keyboard and/or a screen can be used during this phase in order to (at least) install a headless operating
system (Debian or Ubuntu) and execute the required Bootstrap commands.

Bootstrap commands will be detailed in an annex of this documentation.


\section{Eco-classifier}

The eco-classifier module is the main program. It handles image acquisition, conversion, treatment, analysis
and communication with the \gls{plc}.

It's divided in smaller functions that perform individual tasks, triggered by the \gls{plc}.

\subsection{Main loop}

Main loop's purpose is to open necessary connexions, listen to the \gls{plc} and decide what to do depending on what's available here.

Here's the list of tasks this section must provide:

\begin{enumerate}
    \item Handle the \gls{heartbeat} (see \ref{section:heartbeat})
    \item Check if a barcode scan has to be performed (see \ref{section:barcode})
    \item Check if a plastic recognition task has to be performed (see \ref{section:classifier})
    \item Wait for a little while and go back to step 1
\end{enumerate}

\subsection{Barcode reader}

The barcode reader function is triggered by one of the following signals:

\begin{itemize}
	\item Barcode read
	\item Barcode learn
\end{itemize}

The operations for the barcode read sequence are these:

\begin{enumerate}
    \item Open the \gls{vtcamera}
    \item Check that the \gls{plc} command is still "barcode reading", if not skip next step
    \item Read a barcode if present (if not present, return to previous step)
    \item Close the camera
    \item Transmit the barcode value and/or status to the \gls{plc} if we have it
\end{enumerate}

\importantbox{If no barcode can be read, it's up to the \gls{plc} to decide to cut the connexion
by using the command value}

\importantbox{When returning to the PLC, the output value must be ASCII characters (ie. starting from 64)}

EAN length in the \gls{plc} is 32~characters.

For barcode learn, the sequence is:

\begin{enumerate}
    \item Open the \gls{vtcamera}
    \item Check that the \gls{plc} command is still "barcode reading", if not go to close camera
    \item Read a barcode if present (if not present, return to previous step)
	\item Perform a top-facing acquisition with the \gls{vtcamera}
	\item Open the \gls{hzcamera} and light, perform the acqusition and close it
    \item Close the \gls{vtcamera}
    \item Transmit the barcode value and/or status to the \gls{plc} if we have it
	\item Save images to the \texttt{/home/majurca/var/acquisitions} folder
\end{enumerate}

This way, an important number of acquisitions can be made without mechanical interaction from the machine itself.

\section{Bootstrap module}


\section{Main loop}

Main loop's purpose is to open necessary connexions, listen to the \gls{plc} and decide what to do depending on what's available here.

Here's the list of tasks this section must provide:

\begin{enumerate}
    \item Handle the \gls{heartbeat} (see \ref{section:heartbeat})
    \item Check if a barcode scan has to be performed (see \ref{section:barcode})
    \item Check if a plastic recognition task has to be performed (see \ref{section:classifier})
    \item Wait for a little while and go back to step 1
\end{enumerate}

\section{Hearbeat}


\section{Container: \texttt{ecoclassifier-sync}}

\subsection{Container role}

This is the container that syncs acquired images to our Azure BLOB.

\subsection{Dependencies}

\begin{itemize}
    \item Azure Command-Line Interface
\end{itemize}

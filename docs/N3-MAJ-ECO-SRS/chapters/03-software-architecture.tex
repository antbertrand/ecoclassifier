%!TeX encoding = UTF-8
%!TeX root = ../main.tex
\chapter{Architecture (software)}
\label{chapter:softarchitecture}

The architecture is slightly different for each phase of the project, namely:

\begin{itemize}
    \item Image grabbing and capturing (for dataset-building phase)
    \item Deep-Learning computation (for training phase)
    \item Inference (for inference phase)
\end{itemize}

This repository contains all the required code for the three phases, but this code cannot execute
on the same machine at each step.

Some code require hardware parts to run (dataset building and inference need the cameras), some don't
(deep-learning training doesn't require anything else but a powerful GPU).

The following section is thus split in those different phases.

\section{Image grabbing (training)}

The image grabbing part consists in a software piece that reads from (both) cameras and record enough images to
allow a careful analysis of the captured images.

Capture must be as close as possible to the real conditions of the project.

Capture must retain as much elements as possible but not overflow the hard drives.

Capture is made through an OpenCV + Basler software combination. The general flow is:

\begin{enumerate}
    \item Open cameras, start acquiring (without saving on disk)
    \item If "something moves" on camera, start the recording process
    \item Record as much frames as possible to ensure high resolution acquisition
    \item Close camera (and start over)
    \item Upload material on an Azure blob
\end{enumerate}

This piece of software must be installed on a temporary PC that will be plugged to the cameras.

The software must start when the PC starts (to handle reboots)

The acquired images must be either uploaded or accessed remotely. Thus, the PC needs a permanent or semi-permanent internet connexion.

\section{System architecture}

Inside this operating system runs a \gls{docker} container that will contain the main loop of our program.

The installation procedure for the non-docker part should be ultra-trivial.

\subsection{Main system installation}

Here's the general outline for software installation:

\begin{itemize}
    \item Install a bare-naked Linux-based Operating System
    \item Install \gls{docker} and \texttt{docker-compose} on this OS
    \item Install \texttt{supervisord} to manage automatic launch at start-time (see \url{http://supervisord.org/})
    \item Install the bootstrap code
\end{itemize}

\subsection{Bootstrap code installation}

\todo{Document this}


\section{Program architecture}

The main Majurca Ecoclassifier program runs in a PC that's connected to the machine and to the \gls{plc}.

\todo{Decide if the \gls{heartbeat} is in the same thread or in another thread}

It's composed of a few different component/software blocks that all have a specific purpose.

\subsection{Main loop}

Main loop's purpose is to open necessary connexions, listen to the \gls{plc} and decide what to do depending on what's available here.

Here's the list of tasks this section must provide:

\begin{enumerate}
    \item Handle the \gls{heartbeat} (see \ref{section:heartbeat})
    \item Check if a barcode scan has to be performed (see \ref{section:barcode})
    \item Check if a plastic recognition task has to be performed (see \ref{section:classifier})
    \item Wait for a little while and go back to step 1
\end{enumerate}



\section{Maintenance, diagnosis}

\subsection{Automatic restart}

\todo{Decide what's the best policy here}

We have to use either Docker's auto-restart or Supervisor's auto-restart policy here.

For Docker's policies, see \url{https://docs.docker.com/config/containers/start-containers-automatically/}


\subsection{Automatic update}

\todo{Describe here}

% Glossary and acronyms management
% See https://en.wikibooks.org/wiki/LaTeX/Glossary#Defining_symbols
% for more information.
%
% La forma de definir un acrónimo es la siguiente:
% \newacronyn{id}{siglas}{descripción}
% Donde:
% 	'id' es como vas a llamarlo desde el documento.
%	'siglas' son las siglas del acrónimo.
%	'descripción' es el texto que representan las siglas.
%
% Para usarlo en el documento tienes 4 formas:
% pepper your writing with \gls{mylabel} macros (and similar) to simultaneously insert your predefined text and build the associated glossary.
% \glsentryshort{id} - Añade solo las siglas de la id
% \glsentrylong{id} - Añade solo la descripción de la id
% \glsentryfull{id} - Añade tanto  la descripción como las siglas

\newacronym{n3}{N3}{NumeriCube}
\newacronym{cli}{CLI}{Command Line Interface}
\newacronym{srs}{SRS}{Software Requirement Specifications}
\newacronym{plc}{PLC}{Programmable Logic Controller}

\newglossaryentry{heartbeat}
{
    name={heartbeat},
    description={
        A variable that's read from the \gls{plc} and written back every 10~seconds or so.
        If the variable is not updated in this delay, the \gls{plc} declares the camera software as faulty.
    }
}

\newglossaryentry{github}
{
    name={GitHub},
    description={
        The source management service (owned by Microsoft).
        See \url{https://www.github.com}
    }
}

\newglossaryentry{travis}
{
    name={Travis CI},
    description={
        The continuous integration platform.
        See \url{https://www.travis-ci.com}
    }
}

\newglossaryentry{environment}{
    name={environment},
    description={
        An environment is the execution context of a software (either part of it or the whole of it).
        Environments are described clearly in \texttt{dmake} and identified by name.
        An envionment may describe variables, accesses, databases, datasets and specific software
        versions that make a project work. Executed code is \emph{always} run into a specific environment.
        When no environment is explicitely defined, it is assumed that it is the \texttt{develop} environment
        we're talking about.
    }
}

\newglossaryentry{azure}
{
    name={Azure™️},
    description={
        Azure™️ is the cloud platform from Microsoft™️
    }
}

\newglossaryentry{aws}
{
    name={Amazon Web Services™️},
    description={
        Amazon Web Services™️ is the cloud platform from Amazon™️
    }
}

\newglossaryentry{release}
{
    name={release},
    description={
        A release is a state of a specific repository branch at a given date and time that is marked by a tag.
        Release tags are always of the form \texttt{v2019-01-01-123212-branch} where 2019-01-01 is the date
        the tag has been created, 123212 is the time (hours, minutes, seconds) and \emph{branch} is the name
        of the branch in the code repository that's being released.
    }
}

\newglossaryentry{saas}
{
    name={SaaS},
    description={
        SaaS stands for Software as a Service.
        A software compenent that is not executed from a local machine or server
        but accessed through a cloud service provider.
    }
}

\newglossaryentry{draft}
{
    name={Draft},
    description={
        A draft document is a document that is neither production-grade nor
        delivery-grade. A document whose reference is postfixed by the "\-DRAFT"
        mention is a document that has been generated during development process
        with misalignment between the repository revision and the moment it's been
        generated.
    }
}

% \newglossaryentry{real number}
% {
%   name={real number},
%   description={GROUMPF. include both rational numbers, such as $42$ and
%                $\frac{-23}{129}$, and irrational numbers,
%                such as $\pi$ and the square root of two; or,
%                a real number can be given by an infinite decimal
%                representation, such as $2.4871773339\ldots$ where
%                the digits continue in some way; or, the real
%                numbers may be thought of as points on an infinitely
%                long number line},
%   symbol={\ensuremath{\mathbb{R}}}
% }

\newglossaryentry{docker}
{
  name={Docker},
  description={include both rational numbers, such as $42$ and
               $\frac{-23}{129}$, and irrational numbers,
               such as $\pi$ and the square root of two; or,
               a real number can be given by an infinite decimal
               representation, such as $2.4871773339\ldots$ where
               the digits continue in some way; or, the real
               numbers may be thought of as points on an infinitely
               long number line},
  symbol={\ensuremath{\mathbb{R}}}
}

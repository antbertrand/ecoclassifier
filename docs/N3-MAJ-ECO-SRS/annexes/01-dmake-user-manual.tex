%!TeX encoding = UTF-8
%!TeX root = ../main.tex
\chapter{Dmake user manual}
\label{chapter:dmakeman}

This chapter is a basic revamp of \texttt{dmake.py} online documentation.

\section{General information}

\texttt{dmake}\footnotemark is the main entrypoint for Docker+Swarm managed projects.
\footnotetext{This is probably not your favorite name. It just means "Docker make".
Not fancy but does the job.}

\subsection{About dmake}

\todo{Explain the fundamentals behind dmake}

\subsection{Prerequisites}

In order to use dmake, you need:

\begin{itemize}
    \item A POSIX-compliant operating system
    \item Python installed and in the \texttt{PATH}
    \item Access to NumeriCube's \gls{github} repository holding dmake
    \item You must work on a \gls{github}-hosted project (starting with \texttt{git clone})
\end{itemize}

\subsection{Getting started}

You must ensure the prerequisites above are fulfilled.

Before going any further, you must include \texttt{dmake} in your project.

To do so, you must have access to NumeriCube's git repository holding \texttt{dmake}. As of feb 2019, dmake is only available from a GitHub private repo.

Here's the procedure to use dmake on a project:

\begin{itemize}
    \item Go to your project root
    \item \texttt{mkdir ./provision}
    \item \texttt{mkdir ./docs}
    \item Run \texttt{git submodule add https://github.com/numericube/dmake.git ./provision/dmake}
    \item Run \texttt{ln -s ./provision/dmake/dmake.py .}
    \item (optional) Run \texttt{dmake bootstrap -w} \emph{project\_name}
    \item Edit your \texttt{README} file to add the following text:
        \begin{lstlisting}[style=Python-color]
In order to populate the dmake directory, when you clone this project, run the following commands:
$ git submodule init
$ git submodule update
        \end{lstlisting}
    \item You're good to go.
\end{itemize}

\warningbox{When you include a submodule this way, other users of your main github project
will probably find an empty \texttt{provision/dmake} folder when they clone your project.
The \texttt{git submodule init/update} commands will take care of that but don't forget
to put these into your \texttt{README} file.}

\subsection{Basic commands}

See the general philosophy section above to understand how it is working.


\begin{lstlisting}[style=console,caption={Dmake basic usage}]
usage: dmake.py [-h] [-v] [-e ENV] [-m MACHINE] [--aws]
                [--aws-profile AWS_PROFILE] [--aws-region AWS_REGION]
                [--azure]
                {bootstrap,deploy,doc,docker,docker-compose,docker-machine,release,shell,stack,status}
                ...

Handle all commands related to the project: prepare, build, release and maintain it. Are you lost? Start with 'make.py status' or 'make.py status -h' to have hints on how to organize your stack.

positional arguments:
  {bootstrap,deploy,doc,docker,docker-compose,docker-machine,release,shell,stack,status}
                        Action to perform on your source tree
    bootstrap           Bootstrap a new project. Pass along complimentary
                        options to create a project with specific settings.
                        Files are only [OVER]written with the --write option.
    deploy              Deploy a released stack into a production environment.
                        IT WON'T WORK NEITHER ON DEV ENVIRONMENT NOR IF IMAGES
                        HAS NOT BEEN PUSHED
    doc                 Make a new release from the given source tree, ready
                        for Continuous Integration process. If you want to
                        create a new doc, use the following command:
                        dmake doc --create N3-CUS-PRJ-TYP You can see the
                        doc generation progress with dmake -v option.
    docker              Execute docker with the proper arguments. Especially
                        useful with '--machine'. Put your arguments in quotes
                        to have them processed correctly (sorry). If your
                        command starts by a modifier, add a space before it to
                        avoid it being processed by dmake. For example:
                        dmake docker " -help".
    docker-compose      Execute docker-compose with the proper arguments. Pass
                        them in quotes to have them processed correctly
                        (sorry). If your command starts by a modifier, add a
                        space before it to avoid it being processed by
                        dmake. For example: dmake docker-compose "
                        -help".
    docker-machine      Execute docker-machine with the proper arguments. Pass
                        them in quotes to have them processed correctly
                        (sorry). If your command starts by a modifier, add a
                        space before it to avoid it being processed by
                        dmake. For example: dmake docker-machine "
                        -help".
    release             Make a new release from the given source tree, ready
                        for Continuous Integration process.
    shell               Run a sub-shell with all environment variables set
                        according to your project settings
    stack               Manage a (live) docker-compose stack: start, exec,
                        inspect (attach), etc
    status              Double-check that this project structure is up and
                        running.

optional arguments:
  -h, --help            show this help message and exit
  -v, --verbose         Verbose mode
  -e ENV, --env ENV     Environment you're working with (default=dev)
  -m MACHINE, --machine MACHINE
                        Specify a docker-machine name to work on. Use 'make.py
                        status' to get available machines and don't forget the
                        driver argument if necessary.
  --aws                 Run everything with Amazon Web Services (esp. AWS-ECR)
  --aws-profile AWS_PROFILE
                        Use another (non-default) profile
  --aws-region AWS_REGION
                        Specify which region to use
  --azure               Run everything within Azure

\end{lstlisting}

\section{Working locally}

\todo{Explain what happens locally}

\section{Working remotely}

\subsubsection{Release}

\subsection{Inspect a running stack}

Simplest way to check what's running (assuming the machine is registered in your \texttt{docker-machine}
environments):

\begin{lstlisting}[style=console]
dmake --machine <my-machine> status
\end{lstlisting}

Output looks like:

\begin{lstlisting}[style=console]
Docker Swarm Stacks
===================
NAME                SERVICES            ORCHESTRATOR
n3demos-prod        2                   Swarm

'n3demos-prod' stack (docker stack ps n3demos-prod --format 'table {{.Name}}\t{{.Image}}\t{{.CurrentState}}\t{{.Error}}'  -f 'Desired-state=Running' -f 'Desired-state=Ready')
NAME                           IMAGE                                                                      CURRENT STATE         ERROR
n3demos-prod_n3demos-nginx.1   numericube.azurecr.io/numericube/n3demos-nginx:v2019-01-22-111639-master   Running 3 weeks ago
n3demos-prod_n3demos-front.1   numericube.azurecr.io/numericube/n3demos-vue:v2019-01-22-111639-master     Running 3 weeks ago

Active services (docker service ls)
ID                  NAME                         MODE                REPLICAS            IMAGE                                                                      PORTS
yi5y6tizlo08        n3demos-prod_n3demos-front   replicated          1/1                 numericube.azurecr.io/numericube/n3demos-vue:v2019-01-22-111639-master     *:3000->3000/tcp, *:9080->80/tcp
rzqf49o4fmwt        n3demos-prod_n3demos-nginx   replicated          1/1                 numericube.azurecr.io/numericube/n3demos-nginx:v2019-01-22-111639-master   *:80->80/tcp, *:443->443/tcp
\end{lstlisting}
